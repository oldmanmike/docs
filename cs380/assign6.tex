\documentclass{article}
\title{Theory of Computation \\ Assignment 6}
\author{Michael Carpenter}
\date{\today}
\begin{document}
\maketitle

\begin{itemize}
  \item 5.1 - We can use theorem 5.13 and use proof by contradiction. We know that $ALL_{CFG}$ is undecidable. Yet $EQ_{CFG}$ suggests: \\
    $EQ_{CFG}$ = $\{\langle M_{1}, M_{2}\rangle | M_{1}$ and $M_{2}$ are CFGs and $L(M_{1}) = L(M_{2})\}$ to be the case. But say we try to prove this with $ALL_{CFG}$ and say $EQ$ is a TM that decides $EQ_{CFG}$:\\
    S = "On input $\langle EQ,M, w \rangle$\\
    \begin{itemize}
      \item 1. Construct CFGs $G_{1}$ and $G_{2}$
      \item 2. If $M$ accepts $w$, then both $G_{1}$ and $G_{2}$ generate all strings, else each generate no strings.
      \item 3. Run EQ on $G_{1}$ and $G_{2}$.
      \item 4. If EQ accepts, accept. If it rejects, reject."
    \end{itemize}
    This is undecidable because you cannot check equality in a decidable manner on two grammars of $ALL_{CFG}$ as $ALL_{CFG}$ is known to be undecidable.
  \item 5.2 - We can find out if $EQ_{CFG}$ is co-Turing-recognizable by showing $\overline{EQ_{CFG}}$ is recognized by a Turing machine X.\\
    X = "On input $\langle A, B\rangle$,
    \begin{itemize}
      \item 1. Let $s$ be any string from generated by a grammar in $ALL_{CFG}$.
      \item 2. If $s$ is in either $L(A)$ or $L(B)$ but not both, accept. Otherwise, keep going."
    \end{itemize}
  \item 5.3 - One match: $\{[\frac{aa}{a}],[\frac{aa}{a}],[\frac{b}{a}],[\frac{ab}{abab}],[\frac{ab}{abab}],[\frac{aba}{a}]\}$
  \item 5.4 - Not exactly, you could have a regular language $10^{*}1$ for $A$ and a nonregular language $C = \{w|w$ has an equal number of 0s and 1s $\}$ for B. We can take a string in A ("10001" for instance, and reduce it via a reduction that subtracts all 0s except for a number equal to the current number 1s (so it would yield "1001") which is in the nonregular language B. So just because a string $w$ in a regular language A is reducable to a string $w'$ in B doesn't mean B is forced to be
    regular as well.
  \item 5.9 - In problem 4.21 from the assignment 5, we saw that a Turing machine $S$ that accepted $w^{R}$ whenever it accepted $w$ needed to loop over a DFA and mark states that had transitions coming into them from already marked states. This loop can be undecidable if $w$ is not finite. It should be finite, as you cannot reverse an infinite string. But a Turing machine has no way of knowing that and thus no way of knowing when to halt. Thus, the Turing machine $T$ is undecidable via the
    halting problem in theorem 5.1.
  \item 5.23 - We can use theorem 5.22 to show that if $0^{*}1^{*}$ is decidable, then A is decidable. We can make a decider for $0^{*}1^{*}$ called $M$ and a reduction from A to $0^{*}1^{*}$ called $f$. So: \\
    $N$ = "On input $w$:
    \begin{itemize}
      \item 1. Compute $f(w)$
      \item 2. Run $M$ on input $f(w)$ and output whatever $M$ outputs.
    \end{itemize}
\end{itemize}

\end{document}
